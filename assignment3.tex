\documentclass[11pt]{article}
\usepackage{geometry}                % See geometry.pdf to learn the layout options. There are lots.
\geometry{letterpaper}                   % ... or a4paper or a5paper or ... 
\usepackage{graphicx}
\usepackage{amssymb}
\usepackage{epstopdf}


\title{Computer Networking: Assignment 3}
\author{Marzio Lunghi, Dylan Reid Ramelli, Dario Rasic}

\begin{document}
	\maketitle
	\newpage
	\section*{Exercise I}
	Question: How big is the MAC address space? The IPv4 address space? The Ipv6 address space?\\
	$2^{48}$ MAC addresses because it is a 48 bit address space; $2^{32}$ IPv4 addresses; $2^{128}$IPv6 addresses.
	
	\section*{Exercise II}
	Question: Compare the frame structures for 10BASE-T, 100BASE-T,
	and Gigabit Ethernet. How do they differ?\\ \\
	10BASE-T, 100BASE-T and Gigabit Ethernet have identical frame structure.
	The only difference between the three technologies is speed at which they transmit the data.
	The fields in the Ethernet frame structure are as follows:
	\begin{itemize}
		\item The “Data Field” is used to carry IP datagram and size is 46 to 1500 bytes.
		\item 	The “Destination Address Field” contains destination adapter’s MAC address and size is 6 bytes.
		\item 	The “Source Address Field” contains adapter’s MAC address that sends the frame to LAN and size is 6 bytes. 
		\item 	The “Type2 Field” helps the Ethernet to connect with multiplex network – layer protocols and size is 2 bytes.
		\item The “Cycle Redundancy Check (CRC) field”, is used to detect the errors using the CRC field and size is 4 bytes.
		\item 	The “Preamble field” is the first field used to identify the beginning of the Ethernet frame and size is 8 bytes.
	\end{itemize}

	\section*{Exercise III}
	\section*{Exercise IV}
	Question: If all the links in the Internet were to provide reliable delivery service, would the TCP reliable delivery service be redundant? Why or why not?\\ 
	
	TCP Reliable Delivery Service is not redundant. Although each link guarantees that an IP datagram sent over the link will be received at the other end of the link without errors, it is not guaranteed that the IP datagrams will arrive at the ultimate destination in the proper order. With IP, datagrams emerging from the same TCP connection can take different routes in the network, and therefore arrive out of order. TCP is still needed to provide the receiving end of the application the byte stream in the correct order, also, IP can lose packets due to routing loops or equipment failure.
	
	\section*{Exercise V}
	Question: 
	As a mobile node gets farther and farther away from a base station, what are two actions that a base station could take to ensure that the loss probability of a transmitted frame does not increase?\\
	
	The two possible ways are, 1) increasing the transmission power, and 2) reducing the transmission rate.
	\section*{Exercise VI}
	\section*{Exercise VII}	
	\section*{Exercise VIII}
	\section*{Exercise IX}
\end{document}
